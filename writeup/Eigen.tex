\documentclass[]{article}


%%%%%%%%%%%%%%%%%%%%%%%%%%%%%%%%%%%%%%
\usepackage{graphicx}

%%%%%%%%%%%%%%%%%%%%%%%%%%%%%%%%%%%%%%
\newcommand{\abs}[1]{\left| #1 \right|}
\newcommand{\bra}[1]{\left\langle #1 \right|}
\newcommand{\ket}[1]{\left| #1 \right\rangle}
\newcommand{\braket}[2]{\left\langle {#1{\left|\vphantom{#1 #2} \right.} #2} \right\rangle}

\newcommand{\T}{\mathcal{T}}
%%%%%%%%%%%%%%%%%%%%%%%%%%%%%%%%%%%%%%

%%%%%%%%%%%%%%%%%%%%%%%%%%%%%%%%%%%%%%




\title{Systematic Invesitagtion of Optical Eigenmodes in Turbulent ``Free-Space" Channels}
\author{Lukas Scarfe}

\begin{document}
\maketitle

\begin{abstract}
Free space optical systems do not truly operate in free space.
Turbulence in the atmosphere causes spatially and temporally varying phase shifts on the transverse profile of an optical mode throughout the path of propagation.
These phase shifts result in errors when using optical modes for communication.
Recent technologies such as adaptive optics have improved our ability to compensate for such turbulence, however these systems are limited in their ability to correct for higher order aberrations. Recent work 

\end{abstract}

\section{Introduction}

\section{Theory}

\subsection{Hermitian ``Transfer Matrix"}
If we take the transfer matrix, $\T$, and multiply by the complex conjugate, $\T^*$, we obtain a Hermitian matrix, $\T_H$. By finding the eigenvalues of this matrix, instead of those of $\T$, we obtain a different set of modes that are guaranteed to be orthogonal to one another. These modes are not exactly eigenmodes of the channel however. They propagate through the channel and are distinguishable from one another at all times.  

\section{Simulations}

\begin{figure}
    \centering
    \includegraphics[width=\textwidth]{figures/96/96_0Turb_Eigen.png}
    \caption{Eigenmodes of free-space propagation with a resolution of $96\times96$ pixels}
    \label{fig:96_0Turb-Eigen}
\end{figure}


Using a larger simulation window, but reducing the aperature that is actually being used allows for simulations where the boundary conditions are not causing strange errors.


\begin{table}
    \centering
    \begin{tabular}{|c|c|c|}
        \hline
        Number of Pixels & Time to Calculate Eigenvalues & Quality \\ \hline
        32               & 00:00.70                      & Bad     \\ \hline
        48               & 00:04.10                      & Okay    \\ \hline
        64               & 00:17.60                      & Good    \\ \hline
        96               & 02:55:00                      & Good    \\ \hline
        128              &                               & Good    \\ \hline
    \end{tabular}
    \caption{Table of the times it takes to calculate the eigenvalues.}
    \label{T:eigenCalcTime}
\end{table}


\section{Experiment}

\section{Conclusion}



\end{document}