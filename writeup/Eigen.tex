\documentclass[]{article}


%%%%%%%%%%%%%%%%%%%%%%%%%%%%%%%%%%%%%%

%%%%%%%%%%%%%%%%%%%%%%%%%%%%%%%%%%%%%%
\newcommand{\abs}[1]{\left| #1 \right|}
\newcommand{\bra}[1]{\left\langle #1 \right|}
\newcommand{\ket}[1]{\left| #1 \right\rangle}
\newcommand{\braket}[2]{\left\langle {#1{\left|\vphantom{#1 #2} \right.} #2} \right\rangle}
%%%%%%%%%%%%%%%%%%%%%%%%%%%%%%%%%%%%%%

%%%%%%%%%%%%%%%%%%%%%%%%%%%%%%%%%%%%%%




\title{Systematic Invesitagtion of Optical Eigenmodes in Turbulent ``Free-Space" Channels}
\author{Lukas Scarfe}

\begin{document}
\maketitle

\begin{abstract}
Free space optical systems do not truly operate in free space.
Turbulence in the atmosphere causes spatially and temporally varying phase shifts on the transverse profile of an optical mode throughout the path of propagation.
These phase shifts result in errors when using optical modes for communication.
Recent technologies such as adaptive optics have improved our ability to compensate for such turbulence, however these systems are limited in their ability to correct for higher order aberrations. Recent work 

\end{abstract}

\section{Introduction}

\section{Theory}

\section{Simulation}

\section{Experiment}

\section{Conclusion}



\end{document}